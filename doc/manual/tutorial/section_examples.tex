\section{Examples}

In order to give a more concrete understanding, we will give some examples of calibration code.

\subsection{Stereo pair}

Our first example is a stereo pair calibration. Suppose we have ran Bouguet on the two cameras separately for each of 
$n$ images, and the results are stored in \texttt{Calib\_results1.mat} and \texttt{Calib\_results2.mat}.

First, we would read out the results from the result files:

\begin{verbatim}
indices = 1:n;
[imagePoints1, worldPoints1, Q1, data1] = ...
    caliber.io.bouguet('Calib_results1.mat', indices);
[imagePoints2, worldPoints2, Q2, data2] = ...
    caliber.io.bouguet('Calib_results2.mat', indices);
\end{verbatim}

\texttt{data1} and \texttt{data2} are structs, each containing the intrinsic information produced
by Bouguet.

Then, we can start constructing the tree. We begin by constructing a new \texttt{Tree} instance:

\begin{verbatim}
tree = caliber.tree.Tree();
\end{verbatim}

Supposing that we want to put camera1
at the origin, we would add a node for camera1:

\begin{verbatim}

tree.addNode(caliber.node.GeneralNode('camera1', [], data1, {}));
\end{verbatim}

\texttt{caliber.node.GeneralNode} is the default implementation of a Caliber node.
It allows you to specify a list of known and unknown transformations.
The first node is the root of the tree, and thus has no parent or states.

After this first node is added, our tree looks like this:

\centerline{\xymatrix{
\mathrm{camera1}
}}

Now we add camera2:

\begin{verbatim}

tree.addNode(caliber.node.GeneralNode('camera2', 'camera1', data2, {[]}));
\end{verbatim}

Here we have added a node for camera2 as a child of camera1. 
Note that the last
argument is a cell array with a single \texttt{[]} element, 
indicating that the transformation matrix is unknown, 
and that we would
like Caliber to find it. 

The tree now looks like this:

\centerline{\xymatrix{
\mathrm{camera1} \ar[d] \\
\mathrm{camera2}
}}

The last node we have to add is the pointset node:

\begin{verbatim}

tree.addNode(caliber.node.GeneralNode('pointset', 'root', struct(), cell(n)));
\end{verbatim}

Since only cameras have intrinsics, we pass an empty struct for the node's data.
Since the pointset was in a different position for each pair of photographs, its node has $n$ states.

The tree with all the nodes added looks like this:

\centerline{\xymatrix{
\mathrm{camera1} \ar[d] \ar[r] & \mathrm{pointset} \\
\mathrm{camera2}
}}

The last step before solving is to add the observations.

\begin{verbatim}
for i = 1:n
    observation = caliber.observation.IndependentObservation(...
        'camera1', 'pointset', imagePoints1{i}, worldPoints1{i},...
        Q1{i}, Map({'pointset'}, {i}));
    tree.addObservation(observation);
    observation = caliber.observation.IndependentObservation(...
        'camera2', 'pointset', imagePoints2{i}, worldPoints2{i},...
        Q2{i}, Map({'pointset'}, {i}));
    tree.addObservation(observation);
end
\end{verbatim}

We construct an observation object for each pointset state using the appropriate information,
and add those to the system. 
In this case we use \texttt{IndependentObservation}, which can represent Bouguet-style observations.
The last argument to the constructor tells Caliber that the pointset was
in state $i$ when that observation was taken. Note that only the pointset node was explicitly assigned
a state--nodes not explicitly assigned a state are assumed to be in their first state.

Pictorially, we may represent the observations as dashed arrows:

\centerline{\xymatrix{
\mathrm{camera1} \ar[d] \ar[r] & \mathrm{pointset} \\
\mathrm{camera2} \ar@{-->}[ur]
}}

Finally, we need to define the parameters of the system.

\begin{verbatim}
Kparams = logical([1 0 1; 0 1 1; 0 0 0]);
tweakSpec = {'camera1', 'K', Kparams, [];...
          'camera2', 'K', Kparams, [];...
          'camera2', {'r', 't'}, 'all', 1;...
          'pointset', {'r', 't'}, 'all', 'all'}
\end{verbatim}

The tweak argument to the solver tells Caliber which parameters of the node 
can be changed from their supplied values, which we call tweaks.
\begin{enumerate}
	\item The first element in each row indicates which node the tweak refers to.
	\item The second indicates which parameter type(s) can be tweaked.
	\item The third indicates which axes or coefficients can be tweaked.
	\item The last element, if not empty, selects which columns of the tweak data can be changed.
	For extrinsic transformations, each column corresponds to one node state.
\end{enumerate}

Here we have allowed the 
focal length and principal point axis elements of the camera matrix to
to be tweaked, as well as the extrinsic transformation matrices of the second camera and the pointset.

With this, the problem is fully defined, and we can tell Caliber to solve the problem. 

\begin{verbatim}
[initializer, optimizer] = tree.solve(tweakSpec);
\end{verbatim}

For how to retrieve and display results, see Section \ref{disp_results}.
